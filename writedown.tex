% Created 2020-03-21 Sat 15:36
% Intended LaTeX compiler: pdflatex
\documentclass[11pt]{article}
\usepackage[utf8]{inputenc}
\usepackage[T1]{fontenc}
\usepackage{graphicx}
\usepackage{grffile}
\usepackage{longtable}
\usepackage{wrapfig}
\usepackage{rotating}
\usepackage[normalem]{ulem}
\usepackage{amsmath}
\usepackage{textcomp}
\usepackage{amssymb}
\usepackage{capt-of}
\usepackage{hyperref}
\author{Paul Joshy}
\date{\today}
\title{Writedown Documentation}
\hypersetup{
 pdfauthor={Paul Joshy},
 pdftitle={Writedown Documentation},
 pdfkeywords={},
 pdfsubject={},
 pdfcreator={Emacs 26.3 (Org mode 9.2.6)}, 
 pdflang={English}}
\begin{document}

\maketitle
\tableofcontents


\section{Introduction}
\label{sec:org6e6ef8a}

This is a format that I've devised for jotting down pseudocode. I think it is better because
\begin{itemize}
\item It forces us to think like a compiler and stick to rigid rules
\item It's top to bottom and we don't have to go back and edit what we've written
\item It's equally easy to write it down or type it out in a regular computer
\item It is engineered for repetitive output so it's easier for our brain to recognize patterns easily
\end{itemize}


Idk if it's useful or not but if I'm being ambitious I think it can

\begin{itemize}
\item Work as an export format for logging
\item Can automate test cases by checking congruency of diff writedown files
\item Can be used to teach introductory CS to almost anyone without the need for an actual computer
\end{itemize}

\section{Basic rules}
\label{sec:org3a0cf10}

\begin{itemize}
\item All variables are functions
\item All functions should return output even when piped
\item if you're in a function (f prefixed) block or an argument (<>) block, you don't have to give output
\item deferred output (written in o block) should give outputs with variables or expressions. eg \texttt{variable[output]} or \texttt{?[output]}
\item all outputs except in <> should only be from one variable at a time
\item functions have to be rewritten from scratch when they are changed or they can be piped to include changes
\end{itemize}


\section{Syntax}
\label{sec:org8a658b2}
\subsection{Basic output}
\label{sec:org7f589af}

\begin{itemize}
\item Outputs are usually enclosed within square brackets \texttt{[out]}
\item Outputs returned by functions are enclosed inside angular brackets seperated by argumenets \texttt{<arg1[out],arg2[out]>}
\item Outputs returned by constants are enclosed inside double square brackets \texttt{[[const output]]}
\item We can append outputs inside outputs using curly braces \texttt{["Hi, My name is (name)"]}
\end{itemize}

\subsection{Variables and constants}
\label{sec:orgf435fde}

\begin{itemize}
\item Variables are denoted by regular non spaced words eg: \texttt{var1}
\item They can be assigned values using the \texttt{=} sign eg: \texttt{var1 = "val"}
\item constants are denoted by prefixing c before the name eg: \texttt{cpi=3.14}
\end{itemize}

\subsection{if block}
\label{sec:org649d465}

\begin{itemize}
\item An if block expression ends with \texttt{?}
\item it has a true \texttt{t} block and a false \texttt{f} block
\item they both are ended by a \texttt{et} block or a \texttt{ef} block respectively
\end{itemize}

\subsection{functions}
\label{sec:org9aac404}

\begin{itemize}
\item Functions are denoted by prefixing f before the function name eg: \texttt{fprime}
\item They are ended by prefixing e before the function name eg: \texttt{eprime}
\end{itemize}

\subsection{Output block}
\label{sec:orgd7a7f6e}

Outputs blocks are called when deferred output exists. Deferred outputs are outputs that are written long after the expression is written. for eg, calling a function or a loop.
They are called using \texttt{o} and \texttt{eo} blocks 

\subsection{Comments}
\label{sec:org4f3fdf3}

You can use Single line \texttt{\#} comments

\section{Code example}
\label{sec:orgaf3b3b1}

Here is the code for doing the following algorithm

\begin{itemize}
\item Step 1: set loop = 0 and doubler = 0
\item Step 2: Add 1 to loop
\item Step 3: Multiply doubler by two
\item Step 4: if Loop < 4 go to step 2
\item Step 5: Return doubler
\item Step 6: End
\end{itemize}

eg code
\begin{verbatim}
# simple looping function
loop[0], doubler[1]

fforloop<loop,doubler>
loop++
doubler*2
loop<4?
t
=forloop<loop,doubler>
et

f
r<doubler>
ef

eforloop

=forloop<loop[0],doubler[1]>
o
# 1st iteration
loop[2]
doubler[2]
?[true]
# 2nd iteration
loop[2]
doubler[4]
?[true]
# 3rd iteration
loop[3]
doubler[8]
?[true]
# 4th iteration
loop[4]
doubler[16]
?[false]
<doubler[16]>
# 
eo

\end{verbatim}
\end{document}
